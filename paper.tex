\documentclass[journal, a4paper]{IEEEtran}

% some very useful LaTeX packages include:

%\usepackage{cite}      % Written by Donald Arseneau
                        % V1.6 and later of IEEEtran pre-defines the format
                        % of the cite.sty package \cite{} output to follow
                        % that of IEEE. Loading the cite package will
                        % result in citation numbers being automatically
                        % sorted and properly "ranged". i.e.,
                        % [1], [9], [2], [7], [5], [6]
                        % (without using cite.sty)
                        % will become:
                        % [1], [2], [5]--[7], [9] (using cite.sty)
                        % cite.sty's \cite will automatically add leading
                        % space, if needed. Use cite.sty's noadjust option
                        % (cite.sty V3.8 and later) if you want to turn this
                        % off. cite.sty is already installed on most LaTeX
                        % systems. The latest version can be obtained at:
                        % http://www.ctan.org/tex-archive/macros/latex/contrib/supported/cite/

\usepackage{graphicx}   % Written by David Carlisle and Sebastian Rahtz
                        % Required if you want graphics, photos, etc.
                        % graphicx.sty is already installed on most LaTeX
                        % systems. The latest version and documentation can
                        % be obtained at:
                        % http://www.ctan.org/tex-archive/macros/latex/required/graphics/
                        % Another good source of documentation is "Using
                        % Imported Graphics in LaTeX2e" by Keith Reckdahl
                        % which can be found as esplatex.ps and epslatex.pdf
                        % at: http://www.ctan.org/tex-archive/info/

%\usepackage{psfrag}    % Written by Craig Barratt, Michael C. Grant,
                        % and David Carlisle
                        % This package allows you to substitute LaTeX
                        % commands for text in imported EPS graphic files.
                        % In this way, LaTeX symbols can be placed into
                        % graphics that have been generated by other
                        % applications. You must use latex->dvips->ps2pdf
                        % workflow (not direct pdf output from pdflatex) if
                        % you wish to use this capability because it works
                        % via some PostScript tricks. Alternatively, the
                        % graphics could be processed as separate files via
                        % psfrag and dvips, then converted to PDF for
                        % inclusion in the main file which uses pdflatex.
                        % Docs are in "The PSfrag System" by Michael C. Grant
                        % and David Carlisle. There is also some information
                        % about using psfrag in "Using Imported Graphics in
                        % LaTeX2e" by Keith Reckdahl which documents the
                        % graphicx package (see above). The psfrag package
                        % and documentation can be obtained at:
                        % http://www.ctan.org/tex-archive/macros/latex/contrib/supported/psfrag/

%\usepackage{subfigure} % Written by Steven Douglas Cochran
                        % This package makes it easy to put subfigures
                        % in your figures. i.e., "figure 1a and 1b"
                        % Docs are in "Using Imported Graphics in LaTeX2e"
                        % by Keith Reckdahl which also documents the graphicx
                        % package (see above). subfigure.sty is already
                        % installed on most LaTeX systems. The latest version
                        % and documentation can be obtained at:
                        % http://www.ctan.org/tex-archive/macros/latex/contrib/supported/subfigure/

\usepackage{url}        % Written by Donald Arseneau
                        % Provides better support for handling and breaking
                        % URLs. url.sty is already installed on most LaTeX
                        % systems. The latest version can be obtained at:
                        % http://www.ctan.org/tex-archive/macros/latex/contrib/other/misc/
                        % Read the url.sty source comments for usage information.

%\usepackage{stfloats}  % Written by Sigitas Tolusis
                        % Gives LaTeX2e the ability to do double column
                        % floats at the bottom of the page as well as the top.
                        % (e.g., "\begin{figure*}[!b]" is not normally
                        % possible in LaTeX2e). This is an invasive package
                        % which rewrites many portions of the LaTeX2e output
                        % routines. It may not work with other packages that
                        % modify the LaTeX2e output routine and/or with other
                        % versions of LaTeX. The latest version and
                        % documentation can be obtained at:
                        % http://www.ctan.org/tex-archive/macros/latex/contrib/supported/sttools/
                        % Documentation is contained in the stfloats.sty
                        % comments as well as in the presfull.pdf file.
                        % Do not use the stfloats baselinefloat ability as
                        % IEEE does not allow \baselineskip to stretch.
                        % Authors submitting work to the IEEE should note
                        % that IEEE rarely uses double column equations and
                        % that authors should try to avoid such use.
                        % Do not be tempted to use the cuted.sty or
                        % midfloat.sty package (by the same author) as IEEE
                        % does not format its papers in such ways.

\usepackage{amsmath}    % From the American Mathematical Society
                        % A popular package that provides many helpful commands
                        % for dealing with mathematics. Note that the AMSmath
                        % package sets \interdisplaylinepenalty to 10000 thus
                        % preventing page breaks from occurring within multiline
                        % equations. Use:
%\interdisplaylinepenalty=2500
                        % after loading amsmath to restore such page breaks
                        % as IEEEtran.cls normally does. amsmath.sty is already
                        % installed on most LaTeX systems. The latest version
                        % and documentation can be obtained at:
                        % http://www.ctan.org/tex-archive/macros/latex/required/amslatex/math/


\usepackage{amssymb}
\usepackage{mathtools}
\usepackage{physics}

% Other popular packages for formatting tables and equations include:

%\usepackage{array}
% Frank Mittelbach's and David Carlisle's array.sty which improves the
% LaTeX2e array and tabular environments to provide better appearances and
% additional user controls. array.sty is already installed on most systems.
% The latest version and documentation can be obtained at:
% http://www.ctan.org/tex-archive/macros/latex/required/tools/

% V1.6 of IEEEtran contains the IEEEeqnarray family of commands that can
% be used to generate multiline equations as well as matrices, tables, etc.

% Also of notable interest:
% Scott Pakin's eqparbox package for creating (automatically sized) equal
% width boxes. Available:
% http://www.ctan.org/tex-archive/macros/latex/contrib/supported/eqparbox/

% *** Do not adjust lengths that control margins, column widths, etc. ***
% *** Do not use packages that alter fonts (such as pslatex).         ***
% There should be no need to do such things with IEEEtran.cls V1.6 and later.


% Your document starts here!
\begin{document}

% Define document title and author
	\title{Creating forces in chaste}
	\author{Jonathan Miller
	\thanks{Advisor: Dr. Phillip Murray}}
	\markboth{Mathematical Biology: alignment and attractive ends forces}{}
	\maketitle

% Write abstract here
\begin{abstract}
	 Bacilli is the name given to any bacteria shaped like a rod. Chaste (Cancer, Heart and Soft Tissue Environment) has accommodated bacilli bacteria populations by developing an intersecting spheres model to maintain the rod like shape of bacteria during interaction. This \textbf{capsule force} is designed to ensure no capsule overlaps another capsule. As is known Streptobacilli is the name given to bacilli arranged in a chain. Chaste had not developed models to accommodate such chains.
\end{abstract}

% Each section begins with a \section{title} command
\section{Introduction}
	% \PARstart{}{} creates a tall first letter for this first paragraph
	\PARstart{F}{orces} are implemented in Chaste to aid in understanding of natural phenomena. These forces are developed for simulation advantages. Chaste can currently visualise up to three dimensions of simulations with \textit{cell-centre} and vertex \textit{vertex-based} cell populations. It can define different force laws for cell to cell interaction, different cell cycles, different cell proliferative types, to say a few \cite{ChasteCell}. Using an \textit{agile} approach to software development, Chaste develops new projects in directive intense bursts of code development, followed up by refactoring. In a recent code development burst, bacilli shaped cells were developed into working forces of cell cell interactions. This \textbf{capsule force}, $F_c$, operates when two capsules have overlapped by some distance $d$. A repulsive force is applied to the location of any and all capsules with a $d > 0$, meaning the capsules are overlapping by $d$. This $F_c$ only ensure no two capsules will overlap, it does nothing to affecting the topology of the population, namely developing force laws that do two tasks:
	\begin{enumerate}
	    \item Adhesion of capsules at their end points
	    \item Alignment of capsules with angles, $\theta_i < \frac{\pi}{2}$, between any two neighboring capsules, for $i\in P$, $P$ being the population of cells.
	\end{enumerate}
    This report will outline the progress of developing Chaste code to accommodate \textbf{(1)} and \textbf{(2)}. Furthermore, this report will detail visuals relating to development of code. 
% Main Part
\section{Bacterial Capsules}
	Consider a bacterial capsule to be the construction of a cylinder and two hemispheres, such that a capsule is formed in $\mathbb{R}^n$. Let each capsule be $C_i$ and the determining attributes be a length $l_i$ and a radius $r_i$, such that $C_i = l_i+r_i$. This capsule is abstracted to a a problem of sliding spheres on rods and hence is a matter of intersecting spheres. When two capsules interact, say $C_i$ and $C_j$, there is said to be an \textbf{adhesive force} if each capsule, at its end points touches. If this criterion is met, then the capsules will adhere to each other. If the angle, $\theta$, that is the angle found to be between the two capsules, is $\frac{\pi}{2} \leq \theta \leq \pi $, then an \textbf{alignment force} will be applied to each capsule so that over time, the capsules will converge to $\theta = \pi$.
	\begin{figure}
	    \centering
	    \includegraphics{}
	    \caption{Caption}
	    \label{fig:my_label}
	\end{figure}
\section{Forces}

\subsubsection{Capsules}
\subsubsection{Adhesive ends}
\subsubsection{Alignment}
There are four cases to consider.
\begin{enumerate}
    \item Two dimensional alignment
    \item Two dimensional alignment and anti-alignment
    \item Three dimensional alignment
    \item Three dimensional alignment and anti-alignment
\end{enumerate}
\begin{align}
    E   & = -\gamma (t_1\cdot t_2) \\
    F   & = -\nabla E \\
        & = -\gamma \left(\pdv{E}{\theta_1},\pdv{E}{\theta_2}\right)
\end{align}
Such that we seek to find the minimum of $E$.
\section{Methodology}
\section{Results}
\section{Conclusion}
% Now we need a bibliography:
\begin{thebibliography}{5}

	%Each item starts with a \bibitem{reference} command and the details thereafter.
	\bibitem{ChasteCell}
	Chaste (Cancer, Heart and Soft Tissue Environment) Cell-based Chaste: a multiscale computational framework for modelling cell populations
	\url{http://www.cs.ox.ac.uk/chaste/cell_based_index.html}
	\bibitem{MJH06} % Conference paper
	T.~Mayer, H.~Jenkac, and J.~Hagenauer. Turbo base-station cooperation for intercell interference cancellation. {\em IEEE Int. Conf. Commun. (ICC)}, Istanbul, Turkey, pp.~356--361, June 2006.

	\bibitem{Proakis} % Book
	J.~G.~Proakis. {\em Digital Communications}. McGraw-Hill Book Co.,
	New York, USA, 3rd edition, 1995.

	\bibitem{talk} % Web document
	F.~R.~Kschischang. Giving a talk: Guidelines for the Preparation and Presentation of Technical Seminars.
	\url{http://www.comm.toronto.edu/frank/guide/guide.pdf}.

	\bibitem{5}
	IEEE Transactions \LaTeX and Microsoft Word Style Files.
	\url{http://www.ieee.org/web/publications/authors/transjnl/index.html}

\end{thebibliography}

% Your document ends here!
\end{document}